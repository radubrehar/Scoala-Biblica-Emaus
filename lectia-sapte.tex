\newpage

\section*{Lecția 7}

\subsection*{Alegerea unei biserici}

Cum poate şti un creştin cărei biserici să i se alăture? O persoană devine parte a adevăratei Biserici în momentul în care este mântuită. Numele ei este scris în ceruri, în Cartea Vieţii Mielului (Apocalipsa 21:27). Adevărata Biserică este formată din fiecare credincios în Domnul Isus Hristos, indiferent de statutul sau rasa acestuia. Membrii adevăratei Biserici sunt răspândiţi în toată lumea, cu toate că întreaga Biserică nu s-a întâlnit niciodată într-un singur loc.

Cu toate acestea, creştinii care locuiesc aproape unul de celălalt se întâlnesc împreună ca şi membri ai unei biserici, având ca şi ghid Biblia. În zilele de început ale Bisericii, creştinii se întâlneau în propriile lor case (Romani 16:5 şi Filimon 2). Citim că \textit{ei stăruiau în învăţătura apostolilor, în legătura frăţească, în frângerea pâinii şi în rugăciuni} (Fapte 2:42 şi Coloseni 4:15).

Este aşadar clar că Domnul vrea ca cei credincioşi să se întâlnească regulat ca şi membri ai Bisericii. În Evrei 10:25 suntem îndemnaţi să nu lipsim de la întâlnirile cu credincioşii din adunarea locală. Citim astfel: \textit{să nu părăsim adunarea noastră cum au unii obicei} – ceea ce înseamnă că nu trebuie să renunţăm la obiceiul de a ne întâlni împreună, aşa cum obişnuiesc unii.

Uneori pentru un nou convertit este o problemă atunci când trebuie să aleagă un loc în care să se închine împreună cu alţi credincioşi. Aceasta este o problemă reală când într-o zonă sau într-un oraş sunt mai multe biserici. În unele oraşe sunt multe grupuri diferite de creştini cu diferenţe mari în învăţătura lor.

Cum poate şti un creştin cărui grup local ar trebui să i se alăture? În primul rând, el trebuie să se roage cu seriozitate ca să ştie în mod clar gândurile lui Dumnezeu cu privire la acest lucru. El trebuie să afle cum ar trebui să fie o biserică locală citind Cuvântul lui Dumnezeu. El trebuie să se folosească de Cuvântul lui Dumnezeu pentru a cerceta învăţăturile pe care oamenii le-au transmis mai departe (Isaia 8:20). Următoarea listă are menirea de a-l ajuta pe noul credincios să găsească grupul de creştini potrivit:

\begin{enumerate}

	\item Asigură-te că acel grup crede că toată Sfânta Scriptură este Cuvântul lui Dumnezeu insuflat. Toată Biblia ne-a fost oferită prin Duhul Sfânt. El le-a dat oamenilor cuvintele pe care aceştia le-au scris cu mulţi ani în urmă (2 Timotei 3:16 şi 2 Petru 1:21). De aceea spunem că Biblia este Cuvântul inspirat al lui Dumnezeu. (Vezi „Ce ne învaţă Biblia”, Lecţia 1) Unii oameni nu cred că Biblia a fost insuflată de Duhul Sfânt. Ei spun că anumite părţi din ea reprezintă Cuvântul lui Dumnezeu, iar restul, cuvintele oamenilor. Dar nu este adevărat. Toată Biblia este Cuvântul lui Dumnezeu. Noi trebuie să o credem şi să ne supunem adevărului ei. Este autoritatea finală în tot ce ţine de credinţă, aşadar toate lucrurile pe care le facem ar trebui să fie conduse întotdeauna de ea.
	
	\item Asigură-te că acel grup proclamă adevărul despre Persoana lui Hristos. Mulţi oameni sunt gata să spună că Hristos a fost un mare lider, cel mai mare om care a trăit vreodată, ba chiar Îl numesc divin. Dar adevărul măreţ despre binecuvântatul nostru Mântuitor este că El e Dumnezeu (Coloseni 2:9) şi trebuie mărturisit şi recunoscut ca Dumnezeu de către Biserică.
	
	\item Asigură-te că au o învăţătură corectă despre lucrarea lui Hristos. Biblia ne învaţă că Domnul Isus a trăit o viaţă fără păcat, că a ales să moară pe cruce pentru păcatele noastre, că a fost îngropat, că a înviat din morţi şi că S-a înălţat la cer în trupul Lui înviat. Acum stă la dreapta lui Dumnezeu Tatăl (1 Corinteni 15:1-4). Suntem mântuiţi doar prin credinţa în El, iar orice binefacere sau faptă bună pe care o facem nu are niciun rol în mântuirea noastră (Efeseni 2:8,9).
	
	\item Află care este învăţătura acestui grup cu privire la sângele scump al lui Hristos. Nu poate fi iertare de păcat în afara sângelui Lui (Coloseni 1:14 şi Galateni 1:6-9).
	
\end{enumerate}

În completarea acestor patru lucruri, trebuie să te asiguri că biserica locală mărturiseşte atât prin învăţătură cât şi prin fapte următoarele adevăruri importante:

\begin{enumerate}

	\item Hristos este Capul Bisericii (Coloseni 1:18,19 şi Efeseni 1:22,23). Niciun om nu îşi poate însuşi această poziţie. Acolo unde lui Hristos I se oferă locul de cinste ca şi Cap al Bisericii, credincioşii privesc la El şi doar la El pentru călăuzire şi înţelegerea poruncilor Sale. 
	
	\item Toţi credincioşii sunt membri ai trupului lui Hristos (1 Corinteni 12:12,13). Toţi adevăraţii copii ai lui Dumnezeu trebuie să fie bineveniţi în biserica locală. Singurii credincioşi care nu pot lua parte la părtăşia bisericii sunt:
	
	\begin{enumerate}
		
		\item Cei care nu cred şi nu proclamă învăţătura clară a Cuvântului lui Dumnezeu (2 Ioan 10).
		\item Cei care trăiesc în păcat (1 Corinteni 5:13).
	\end{enumerate}
	
Aceşti oameni pot asista la programele bisericii dar ei nu se pot uni pentru a lua parte la părtăşia credincioşilor.
	
	\item Niciun necredincios nu trebuie primit la părtăşia bisericii.
	
	\item Toţi credincioşii sunt preoţi (1 Petru 2:5,9). În Vechiul Testament exista o diferenţă între preoţi şi restul oamenilor. Oamenii aduceau jertfe lui Dumnezeu dar nu li se permitea să le aducă în prezenţa Lui. Ei ofereau jertfele preoţilor care aveau slujba de a le aduce înaintea lui Dumnezeu. Acest lucru s-a schimbat de la moartea lui Hristos, iar în Noul Testament nu este nicio diferenţă între preoţi şi oameni. Toţi credincioşii sunt preoţi şi pot intra acum prin credinţă în prezenţa lui Dumnezeu. Jertfele pe care ei le aduc lui Dumnezeu sunt închinarea, lauda şi slujirea. Noul Testament ne învaţă că toţi credincioşii ar trebui să studieze Cuvântul lui Dumnezeu, să fie ocupaţi pentru Domnul şi să spună Vestea Bună a Evangheliei oamenilor pe care îi întâlnesc
	
	\item Autoritatea Duhului Sfânt trebuie recunoscută. Asta înseamnă că Duhul Sfânt este Liderul şi El ar trebui să conducă închinarea, slujirea, învăţătura şi disciplina în biserică. Călăuzirea Lui nu trebuie limitată de ideile oamenilor sau de alte ritualuri (2 Corinteni 3:17 şi Efeseni 4:3).
	
\end{enumerate}

Aşadar, pentru a recapitula ce am studiat, un credincios trebuie să caute o biserică locală în care credincioşii folosesc Biblia ca singura lor călăuză, care cred toate învăţăturile Bibliei cu privire la Persoana şi lucrarea lui Hristos şi încearcă să pună în practică învăţăturile Noului Testament cu privire la Biserică şi scopul ei. Credincioşii care sunt de acord cu toate aceste lucruri au ceea ce Biblia numeşte „părtăşie” unii cu alţii (Fapte 2:42).

