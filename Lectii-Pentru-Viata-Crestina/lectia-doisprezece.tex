\newpage

\lectia{12}{Trăind o viaţă cu folos pentru Dumnezeu}

Cum poate trăi un creştin o viaţă cu folos pentru Dumnezeu? De îndată ce o persoană este mântuită, ea este pregătită pentru cer. Nimic din ceea ce face după ce este mântuită nu poate adăuga ceva la mântuirea sa. Dumnezeu spune că noi suntem mântuiţi prin credinţa în Hristos şi nu prin faptele noastre. (Efeseni 2:8,9 şi Tit 3:5). Dacă un creştin este pregătit pentru cer când îşi pune încrederea în Domnul, de ce nu-l ia Domnul în acel moment? Dumnezeu are cu siguranţă un scop pentru care îl lasă pe pământ.

Epistolele Noului Testament, începând cu Romani şi continuând cu celelalte până la Iuda, sunt scrisori scrise credincioşilor. Aceste scrisori vorbesc mult despre cum trebuie să trăiască un creştin. Dumnezeu îi lasă pe creştini pe pământ pentru a arăta felul în care El poate schimba vieţi. Credincioşii sunt dovada harului mântuitor al lui Dumnezeu. (Vezi ,,Ce ne învaţă Biblia'', Lecţia 9). Privind la vieţile transformate ale creştinilor, oamenii nemântuiţi pot vedea ce a făcut harul lui Dumnezeu. De exemplu, un om care era hoţ este mântuit. El nu mai fură de la oameni ci trăieşte o viaţă curată şi lucrează pentru a înapoia lucrurile pe care le-a furat. Viaţa lui este transformată complet, iar când le spune oamenilor că este născut din nou, ei trebuie să recunoască faptul că în el s-a petrecut o mare schimbare.  

Credincioşii sunt numiţi ogorul lui Dumnezeu (1 Corinteni 3:9), iar Dumnezeu vrea roadă din ogorul Său. În Galateni 5:22,23 citim că roada Duhului (Duhului Sfânt) este dragostea, bucuria, pacea şi aşa mai departe. Roada nu creşte într-o zi, ci e nevoie de timp ca să se dezvolte. Deci Dumnezeu lucrează în viaţa credinciosului zi de zi, pentru a dezvolta în el roada Duhului. Scopul Lui este ca credincioşii să devină asemenea lui Hristos (Romani 8:29). \textit{Asemenea Chipului Fiului Său} înseamnă a fi asemenea lui Hristos. 

       O viaţă cu folos pentru Dumnezeu este o viaţă în care se vede roada Duhului Sfânt (Galateni 5:22,23), o viaţă care este transformată în asemănarea cu Domnului Isus (Romani 8:29). Iată câteva moduri prin care Biblia le spune credincioşilor cum trebuie să trăiască pentru Hristos:
       
	\begin{enumerate}
       
       \item Fii un adevărat urmaş al lui Hristos. Toţi credincioşii sunt copii ai lui Dumnezeu, dar nu toţi sunt adevăraţi urmaşi sau ucenici. Domnul Isus le-a spus ucenicilor ce înseamnă să fi un adevărat ucenic (Matei 10:16-42 şi Luca 14:25-35). A fi un adevărat ucenic înseamnă a-L urma pe Domnul Isus şi a renunţa, a te lepăda de tot ce ai vrea să păstrezi pentru tine. Asta înseamnă să trăieşti o viaţă de renunţare la sine, o viaţă în care renunţi la dorinţele tale egoiste. Deseori oamenilor din jurul nostru nu le place asta şi îşi bat joc de creştini.
       
       \item Predă-ţi viaţa Domnului (Romani 12:1). Acesta este un real act de închinare înaintea lui Dumnezeu. Aceasta este ceea ce ar trebui să facă credinciosul, întrucât Domnul Isus a făcut atât de multe pentru el. Biblia numeşte aceasta o \textit{slujbă duhovnicească} (Romani 12:1). Un credincios a spus odată: ,,Dacă Domnul Isus este Dumnezeu şi a murit pentru mine, atunci niciun sacrificiu făcut de dragul Lui nu poate fi prea mare pentru mine.''
       
       \item Renunţă la planurile pe care le ai cu privire la viaţa ta. Mântuitorul a spus: \textit{,,Cine îşi va pierde viaţa pentru mine, o va câştiga''} (Matei 16:25). Cu alte cuvinte, dacă vrei să cunoşti adevărata bucurie şi fericire a vieţii, trebuie să trăieşti fiindu-I plăcut Domnului Isus Hristos şi nu ţie însuţi. O persoană care trăieşte într-un mod egoist este nefericită.
       
       \item Rupe legăturile pe care le ai cu păcatul; distruge orice lucru care te-ar putea tenta sau ispiti să păcătuieşti. Dă-ţi toate silinţele pentru a întoarce spatele păcatului. Izgoneşte ispita şi să nu ai nimic de a face cu vechea ta viaţă păcătoasă. Citeşte Efeseni 4:22-32. Aici ni se spune să ne lăsăm de lucrurile păcătoase pe care obişnuiam să le facem. Păcatele pe care obişnuiam să le înfăptuim au devenit odioase şi ne bucurăm \textit{să ne dezbrăcăm de ele} (versetul 22). Astfel vom putea trăi o viaţă caracterizată de ascultare şi de dragoste faţă de Dumnezeu. Asta înseamnă să fi un adevărat ucenic al Lui (Luca 9:23).
       
       \item Nu te îndepărta. Mulţi creştini încep bine, dar mai târziu alunecă din nou în vechiul lor mod de trăire. O slujbă bună poate lua uneori prea mult din timpul creştinului şi prin urmare el devine mai nepăsător în citirea Cuvântului lui Dumnezeu şi în rugăciune. Când se întâmplă asta, credinciosul alunecă din nou în vechile lui căi şi cu greu îşi dă seama de ceea ce face. Unii s-au căsătorit fără a căuta voia lui Dumnezeu, deşi a fost un timp când ei doreau cu adevărat să Îl urmeze pe Domnul.  Aceasta i-a făcut să se întoarcă pe căile lor vechi şi păcătoase.
       
       \item Trăieşte pentru a-i sluji pe alţii. \textit{Fiul Omului nu a venit ca să I Se slujească, ci El să slujească} (Matei 20:28). Adevărata măreţie constă în a-i sluji pe alţii. Nu fi unul care mereu vrea să primească. \textit{Este mai ferice să dai decât să primeşti} (Fapte 20:35). 
       
       \item 7.	Fă-L pe Isus Domnul sau Conducătorul vieţii tale. Dacă Hristos conduce viaţa ta, vei avea bucurie acum şi în eternitate. 
       
    \end{enumerate}
    
Ne-ar plăcea ca la sfârşitul acestui curs prin corespondenţă studentul să înţeleagă că viaţa de credinţă nu este o viaţă uşoară, ci o luptă. Nu te costă nimic să devi creştin, dar te costă totul ca să fii unul. Cu toate astea, viaţa de creştin este cea mai bună viaţă. Îi slujieşti celui mai bun dintre stăpâni, iar răsplăţile sunt mari aici, dar şi în veşnicie.

Te îndemnăm ca şi credincios să-I oferi lui Hristos viaţa ta. Dă-I Lui ce ai mai bun. Nu păstra nimic pentru tine. Atunci când Îl vei vedea, fie ca marea ta bucurie să fie aceea de a-L auzi spunând: \textit{Bine, rob bun şi credincios...intră în bucuria Stăpânului tău} (Matei 25:21).
