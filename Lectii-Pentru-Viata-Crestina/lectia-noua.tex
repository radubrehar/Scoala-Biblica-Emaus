\newpage

\lectia{9}{Rugăciunea}

Ce ne învaţă Biblia cu privire la rugăciune? Fără rugăciune nu poate exista progres în niciun domeniu al  vieţii creştine. Este foarte important pentru cel nou în credinţă să cunoască învăţătura Bibliei cu privire la rugăciune. În continuare vom prezenta răspunsuri la câteva întrebări generale despre rugăciune.

\begin{enumerate}

	\item \textit{De ce să ne rugăm?} Pentru că Biblia ne porunceşte să facem asta (1 Timotei 2:8). Domnul Isus a fost un om al rugăciunii. El a simţit nevoia să se roage, iar noi avem nevoie de asta mai mult (1 Tesaloniceni 5:17, 18 şi Efeseni 6:18). 
	
	\item \textit{Cât de des ar trebui să ne rugăm?} Ar trebui să ne rugăm la un timp regulat în fiecare zi, dar şi între acele momente stabilite (Daniel 6:10 şi Psalmul 5:3). Este bine să ne rugăm când ne trezim dimineaţa şi seara înainte de a merge la culcare. Apoi, ar trebui să privim la Domnul pe parcursul zilei, când avem probleme, când avem nevoie de ajutorul Lui sau când vrem să-I mulţumim pentru ceva anume. Orice creştin trebuie să-şi plece capul şi să-I mulţumească lui Dumnezeu înainte de fiecare masă. Ar trebui să facă asta când este singur dar şi când este cu alţi oameni.
	
	\item \textit{În ce poziţie ar trebui să ne rugăm?} Atunci când se ruga, Daniel îngenunchea (Daniel 6:10). Aşa făcea şi Domnul Isus (Luca 22:41). Pe de altă parte, Neemia s-a rugat în timp ce stătea înaintea împăratului (Neemia 2:4). De obicei, creştinii îngenunchează când sunt acasă, dar ei pot vorbi cu Dumnezeu oricând, fără să îngenuncheze. Aceasta se poate întâmpla când sunt ocupaţi la muncă sau în timp ce merg pe stradă. 
	
	\item \textit{Pentru cine sau pentru ce să ne rugăm?} Câteva din versetele care ne răspund la această întrebare se găsesc în Filipeni 4:6, 1 Timotei 2:1-3 şi Matei 9:38. Nu există vreun lucru prea mic sau prea mare pentru a-l transforma într-un motiv de rugăciune. Mulţi credincioşi se folosesc de o listă de rugăciune pe care notează lucruri ca:
	
	\begin{enumerate}

		\item numele celor nemântuiţi din familiile lor sau numele prietenilor nemântuiţi
		\item numele oamenilor bolnavi sau în nevoie
		\item numele oamenilor care Îl slujesc pe Domnul, cum sunt misionari, păstori, învăţători şi alţii.

	\end{enumerate}
	
	Vei vedea cum Domnul răspunde rugăciunilor tale atunci când te rogi pentru oameni pe nume. Vei vedea cum Domnul lucrează prin răspunsurile la rugăciunile tale pe măsură ce aduci înaintea Lui o nevoie specifică.
	
	\item \textit{Cum trebuie să ne rugăm ca rugăciunile noastre să primească răspuns?}
	
	\begin{enumerate}
	
		\item Rugăciunile noastre vor fi ascultate dacă rămânem în Hristos (Ioan 15:7). Ce înseamnă să rămâi în Hristos? Atunci când suntem mântuiţi suntem făcuţi una cu Hristos. Unirea noastră cu Hristos nu poate fi distrusă, iar mântuirea noastră este sigură. Însă Ioan 15:7 se referă la desfătarea noastră zilnică în Hristos. Aceasta se numeşte părtăşie sau comuniune cu Hristos. Ea continuă zi după zi pe măsură ce citim Scriptura şi ascultăm ce scrie în ea. Dacă neglijăm citirea Cuvântului lui Dumnezeu sau nu ne supunem la ceea ce El ne-a scris, părtăşia cu Dumnezeu este întreruptă. Suntem în continuare mântuiţi, dar nu rămânem în Hristos. A rămâne în Hristos înseamnă a păzi poruncile Lui şi a ne bucura mereu de părtăşia cu El (1 Ioan 3:22).
		
		\item Rugăciunile noastre trebuie să fie în acord cu voia Lui (1 Ioan 5:14). Când rămânem în Hristos cunoaştem care sunt lucrurile care reprezintă voia lui Dumnezeu. Pe măsură ce citim Biblia învăţăm tot mai mult voia lui Dumnezeu, iar rugăciunile noastre sunt în acord cu aceasta. 
		
		\item Rugăciunile noastre trebuie să fie în numele lui Hristos (Ioan 14:13 şi 16:23). Gândurile Lui devin gândurile noastre, astfel că atunci când ne rugăm cu adevărat în numele Lui este ca şi cum Domnul Isus ar vorbi cu Dumnezeu. Tatăl aude vocea noastră tot aşa cum aude vocea Fiului Său.
		
		\item Motivul pentru care ne rugăm trebuie să fie unul plăcut Lui (Iacov 4:3). Nu ne putem aştepta la răspuns dacă ne rugăm pentru motive egoiste.
		
	\end{enumerate}
	
	\item \textit{Care sunt avertizările cu privire la rugăciune?}

	\begin{enumerate}
	
		\item Nu vă rugaţi ca să fiţi văzuţi sau lăudaţi de oameni (Matei 6:5,6).
		\item Nu-I cere lui Dumnezeu să facă ceva ce poţi face tu însuţi. Niciun creştin în toate facultăţile mintale nu va păşi înaintea unei maşini după care se va ruga lui Dumnezeu ca să-l aducă înapoi pe marginea drumului. Dumnezeu i-a dat picioare ca să meargă singur înapoi.
		\item Nu cere ceva ce ştii că nu ar trebui să ceri! Uneori Dumnezeu ne dă ceea ce am cerut, dar apoi ne dăm seama că am pierdut mai mult decât am câştigat. Citeşte Numeri 11:4-34 ca să vezi cum a învăţat Israel lecţia aceasta. În Psalmi 106:15 citim: „El le-a dat ce cereau; dar a trimis o molimă printre ei.”
		\item Nu spune cuvinte fără să te gândeşti la semnificaţia lor. Nu rosti din nou şi din nou aceleaşi lucruri (Matei 6:7 şi Eclesiastul 5:2).
		
	\end{enumerate}

	\item \textit{Alte sugestii}
	
	\begin{enumerate}
	
		\item Încearcă să te rogi cu voce tare dacă ţi se pare greu să te concentrezi la rugăciune. Aceasta te va ajuta să te gândeşti la lucrurile pentru care te rogi.
		\item Nu fi descurajat dacă răspunsul nu vine imediat. Răspunsurile lui Dumnezeu nu vin niciodată prea devreme, ca noi să cunoaştem bucuria aşteptării Lui şi nu vin niciodată prea târziu, ca noi să cunoaştem că El e un Dumnezeu în care ne putem încrede.
		\item Dacă răspunsul lui Dumnezeu nu este exact ceea ce ai cerut, aminteşte-ţi că Dumnezeu are dreptul să ne dea ceva mai bun decât am cerut. Noi nu ştim ce este bun pentru noi, dar Dumnezeu ştie şi răspunde în cel mai bun mod (2 Corinteni 12:8,9). Ba chiar ne dă mai mult decât am cerut sau ne-am gândit să cerem vreodată (Efeseni 3:20,21).

	\end{enumerate}
	
\end{enumerate}