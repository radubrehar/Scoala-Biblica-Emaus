
\subsection*{Instrucţiuni pentru cursanți}

Acest curs a fost scris pentru noii credincioşi în Domnul Isus Hristos. Scopul lui este de a-i ajuta pe cei care îşi doresc să afle cum pot creşte în viaţa de credinţă. Ea conţine multe versete, care trebuie căutate şi citite cu atenţie. Acest lucru este foarte important, de vreme ce toate lecţiile sunt bazate pe versete din Scriptură. Versetele biblice citate sunt scrise în italic şi sunt din versiunea Dumitru Cornilescu.

\subsubsection*{Lecțiile pe care le vei studia}

\begin{enumerate}
\itemsep0em 
	\item Consecințele nașterii din nou
	\item O mântuire sigură
	\item Mântuit pentru totdeauna
	\item Biruință asupra ispitei
	\item Comportamentul creștinului
	\item Îngropat în botez
	\item Alegerea unei biserici
	\item Planul lui Dumnezeu pentru viaţa noastră
	\item Rugăciunea
	\item Conducându-i pe alţii la Hristos
	\item Studierea Bibliei
	\item Trăind o viaţă cu folos pentru Dumnezeu
\end{enumerate}


\subsubsection*{Cum să studiezi}

Cere-I lui Dumnezeu să-ţi deschidă inima ca să primeşti adevărurile pe care le vei învăţa din Cuvântul Lui. Citeşte lecţia în întregime pentru a avea o înţelegere de ansamblu. Apoi citeşte din nou, pe îndelete, de această dată asigurându-te că vei căuta toate versetele biblice. Îţi vor fi de folos un caiet şi un creion pentru a nota orice întrebare pe care vei dori să o pui mai târziu. Pune-ţi deoparte în fiecare zi un timp regulat în vederea studiului şi începe să studiezi imediat.

\subsubsection*{Testele de auto-evaluare}

Aceste teste apar la sfârşitul fiecărei lecţii şi trebuie completate înainte de a trece la următoarea lecţie. Testele sunt menite să te ajute să-ţi aminteşti lucrurile pe care le-ai studiat. Ele sunt doar pentru tine – nu trebuie să trimiţi răspunsurile la biroul Emaus.

Unele teste conţin afirmaţii din care au fost scoase cuvinte importante. În spaţiile libere trebuie să notezi cuvântul sau cuvintele corecte.

Unele propoziţii nu sunt finalizate, iar tu va trebui să le completezi alegând finalul potrivit din lista dată, apoi să notezi litera corectă în spaţiul gol de pe marginea din dreapta.

Pentru a verifica dacă răspunsul tău este corect, uită-te la răspunsul sau răspunsurile corecte din partea dreaptă a paginii. Ele apar pe acelaşi rând cu următoarea întrebare. Acoperă răspunsurile atunci când începi testul. Te poţi uita să vezi dacă un răspuns este corect după ce vei fi terminat o întrebare. Corectează-ţi răspunsul dacă este greşit şi întoarce-te la punctul care tratează acest subiect din lecţie.

\subsubsection*{Testele de verificare}

Fiecare test este pentru două lecţii. De exemplu, testul numărul 1 este pentru lecţia 1 şi lecţia 2.

Fiecare test este marcat clar pentru a-ţi arăta întrebările corespunzătoare fiecărei lecţii. Poţi face testul în două părţi. Când ai terminat de studiat lecţia 1, poţi rezolva partea testului corespunzătoare acestei lecţii.

Nu răspunde la întrebări în funcţie de ceea ce crezi sau de ceea ce ai crezut mereu. Întrebările din teste sunt date pentru a verifica dacă înţelegi versetele biblice care sunt date în lecţie.

Atunci când studiezi, poţi folosi orice versiune a Bibliei, dar când răspunzi la întrebări, te rugăm să foloseşti versiunea Cornilescu. Aceasta este versiunea folosită în aceste lecţii.

\begin{enumerate}
	\item \textbf{Întrebările ,,Ce spui tu?''}
	
Aceste întrebări se găsesc la sfârşitul fiecărui test de verificare. Poţi răspunde la ele aşa cum doreşti. Ele nu vor fi considerate ca parte din nota corespunzătoare testului. Eşti liber să spui ce crezi atunci când îţi este pusă această întrebare. Dacă răspunzi sincer, profesorul tău te va cunoaşte mai bine.

	\item \textbf{Cum sunt notate testele}.

Profesorul tău va marca orice răspuns greşit. Îţi va arăta locul din Biblie sau din lecţie unde vei putea găsi răspunsul corect.

\end{enumerate}

\subsubsection*{Grupe și clase}

Dacă studiezi într-o clasă împreună cu alte persoane, predă testul de verificare liderului clasei. El le va trimite pentru tine şi pentru alţii.

\subsubsection*{Instrucțiuni generale}

Începe studiul imediat. Dacă eşti în clasă, începe de îndată ce ora a început. Încearcă să studiezi la un timp regulat în fiecare zi. Ai la dispoziţie un an ca să completezi lecţiile. Încearcă să termini lecţiile cât de repede este posibil. O idee bună este să studiezi câte o lecţie pe săptămână şi să completezi un test de verificare la fiecare două săptămâni.

\vspace*{2cm}
\nota{Notă: Citeşte întotdeauna toate răspunsurile pentru o întrebare înainte de a alege răspunsul pe care îl consideri corect.}
