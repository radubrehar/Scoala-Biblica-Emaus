\newpage

\section*{Lecția 3}

\subsection*{Mântuit pentru totdeuna}

Se poate ca o persoană mântuită să îşi piardă vreodată mântuirea? O persoană care a fost cu adevărat născută din nou nu mai poate să fie pierdută. Ea este mântuită pentru totdeauna. Studiază cu atenţie fiecare din următoarele şapte versete biblice arătate aici. Ele ne învaţă acest adevăr măreţ.

\begin{enumerate}

	\item Ioan 10:27-29. Acestea sunt cuvintele Domnului Isus, oile Mele ascultă glasul Meu. Domnul Isus îi numeşte pe ai Lui oile Sale. În versetul 26, El spune că cei care nu cred nu sunt oile Sale. El Îşi cunoaşte fiecare oaie şi îi oferă fiecăreia în parte viaţa veşnică. În versetul 28, El spune „Eu le dau viaţa veşnică, în veac nu vor pieri...”. Viaţa veşnică înseamnă viaţă pentru totdeauna şi niciunul din cei care au această viaţă nu o pot pierde. Credinciosul poate fi sigur de acest lucru. Aceasta este o promisiune făcută de Domnul Isus şi care nu depinde de creştin. Nimeni nu te poate smulge din mâna Domnului Isus şi nimeni nu te poate smulge din mâna Tatălui Său (versetele 28,29). Atât Domnul Isus cât şi Tatăl te ţin în siguranţă.
	
	\item 2.	Ioan 5:24. Acestea sunt tot cuvintele Domnului Isus. El promite viaţă veşnică celui care aude Cuvântul Său şi crede în Cel ce L-a trimis. Cel care crede în Domnul Isus Hristos nu va fi judecat pentru păcatul lui, ci trece din moarte la viaţă. Fiul lui Dumnezeu ar eşua în a-Şi ţine promisiunea dacă unul din credincioşi şi-ar pierde mântuirea.
	
	\item Ioan 3:36. „Cine crede în Fiul are viaţa veşnică.” Asta înseamnă că un credincios are viaţa veşnică chiar acum. Isus nu spune că el trebuie să aştepte ca să primească viaţa veşnică. Ea îi este dată credinciosului atunci când acesta este mântuit şi nimeni nu i-o poate lua.
	
	\item Romani 8:38,39. Apostolul Pavel ne spune că nici moartea şi nici viaţa nu-l pot despărţi pe credincios de dragostea lui Dumnezeu. Dragostea lui Dumnezeu nu poate fi îndepărtată de îngeri sau de stăpânirile şi puterile cereşti. Nu putem fi despărţiţi de dragostea lui Dumnezeu nici de lucrurile care ni se întâmplă acum, nici de lucrurile care ni se vor întâmpla la un moment dat în viitor. Pavel încheie spunându-ne că nu există nimic în întreaga creaţie a lui Dumnezeu care ar putea fi în stare să ne despartă vreodată de dragostea lui Dumnezeu. Este posibil ca un copil al lui Dumnezeu să păcătuiască, dar păcatul pe care l-a săvârşit nu îl va despărţi de dragostea lui Dumnezeu. Păcatul Îl întristează pe Tatăl ceresc şi îi aduce multă întristare şi credinciosului, dar este un lucru pe care nu-l poate face – nu-i poate lua mântuirea. 
	
	\item 2 Timotei 1:12. Această epistolă a fost scrisă de Pavel când suferea în închisoare pentru că era creştin. Pavel a spus că lui nu îi este ruşine că este întemniţat pentru că ştie în cine s-a încrezut. Fiecare creştin ar trebui să aibă aceeaşi încredere în Domnul. În Ioan 6:39 ni se spune că toţi credincioşii au fost dăruiţi de către Dumnezeu Tatăl lui Hristos, ca El să-i păstreze în siguranţă. Hristos va face cu credincioşie ceea ce Tatăl i-a cerut să facă.
	
	\item Iuda 24. Domnul Isus Hristos este Cel care are putere să-i păzească pe creştini de orice cădere în păcat. Tot El este Cel care îi va duce în siguranţă acasă, în ceruri. Creştinii nu pot merge pe calea mânturii prin propriile forţe, tot aşa cum nu sunt capabili nici să se mântuiască singuri (1 Petru 1:5).
	
	\item Romani 8:30. ...şi pe aceia pe care i-a chemat, i-a şi socotit neprihăniţi; iar pe aceia pe care i-a socotit neprihăniţi, i-a şi proslăvit. Creştinii sunt socotiţi neprihăniţi (justificaţi) când sunt mântuiţi (vezi Lecţia 1, paragraful 3). Fiecare persoană mântuită este deja proslăvită pentru că este justificată. Noi încă nu avem trupuri de slavă (1 Corinteni 15:35-37). Aceasta va avea loc atunci când toţi credincioşii vor fi în prezenţa lui Dumnezeu. În acea clipă trupurile lor vor fi schimbate asemenea trupului lui Hristos după înviere. Hristos este proslăvit acum, iar toţi credincioşii vor fi proslăviţi împreună cu El (Romani 8:17).
	
\end{enumerate}


Amintiţi-vă aceste lucruri şi veţi fi siguri de mântuirea voastră.

\begin{enumerate}

	\item Un creştin nu îşi pierde mântuirea când păcătuieşte. Hristos a plătit deja pedeapsa pentru toate păcatele sale – păcate din trecut, prezent şi viitor. Dumnezeu nu va cere o altă plată pentru păcat. Creştinul nu va trebui să moară pentru păcatele sale pentru că Hristos a murit pentru ele. Dumnezeu este un Judecător drept şi îl iartă pe păcătosul care crede în Hristos.
	
	\item Când un creştin păcătuieşte, Îl întristează pe Dumnezeu, Tatăl lui ceresc. Sentimentul plăcut de apartenenţă la familie dintre un copil şi Tatăl lui se rupe în acel moment. Creştinul rămâne parte din familia lui Dumnezeu, dar este nefericit. Dumnezeu îl iartă pe creştin atunci când acesta îşi mărturiseşte păcatul. Când îşi mărturiseşte păcatul, creştinul Îi spune lui Dumnezeu toate greşelile sale şi recunoaşte că aceste greşeli sunt păcate (1 Ioan 1:9). Domnul Isus Se roagă, mijloceşte pentru creştinul care păcătuieşte (1 Ioan 2:1,2). Domnul Isus este jertfa pentru păcatele creştinului, iar Dumnezeu iartă păcatul datorită jertfei Fiului Său. Creştinul care îşi mărturiseşte păcatul ştie că Dumnezeu l-a iertat şi l-a curăţit de păcatul lui. Tristeţea care a apărut în urma păcatului dispare, iar sentimentul plăcut de apartenenţă la familie dintre Dumnezeu Tatăl şi el revine. 
	
	\item Uneori un creştin păcătuieşte făcând lucruri de care este conştient că sunt greşite. Un copil al lui Dumnezeu care face asta va fi pedepsit de Dumnezeu, Tatăl său (Evrei 12:6,7). Tatăl poate folosi întristarea adâncă sau suferinţa pentru a-l aduce înapoi la El. În unele cazuri, Dumnezeu îi poate scurta viaţa pe pământ: trupul lui moare, dar sufletul lui este mântuit (1 Corinteni 5:5).
	
	\item Un creştin poate păcătui şi o face, dar niciun creştin adevărat nu continuă să facă acelaşi păcat mereu şi mereu. Păcatul îl face foarte nefericit, iar creştinul îl urăşte. Uneori un creştin poate refuza să îşi mărturisească păcatul lui Dumnezeu, continuând să păcătuiască. Apoi Dumnezeu aduce pedeapsa în viaţa lui. Dumnezeu este Tatăl lui şi El îşi pedepseşte copilul care păcătuieşte tot aşa cum un tată pământesc îşi pedepseşte copilul care este neascultător. Totuşi, noi nu ne pierdem mântuirea atunci când păcătuim. Unii oameni cred că pot continua să păcătuiască fără să li se întâmple nimic, dar Dumnezeu vede fiecare păcat, iar noi nu ne putem ascunde de El (Evrei 4:13). Dumnezeu va aduce probleme şi suferinţă în viaţa noastră dacă nu ne mărturisim păcatele. Dumnezeu vrea ca noi să dorim să părăsim păcatul şi să ne întoarcem din nou la El.
	
	\item Unii oameni cred că există versete în Scriptură care spun că un creştin îşi poate pierde mântuirea. Ar trebui să studiezi cu mare atenţie aceste versete şi să citeşti versetele care sunt înainte şi după ele. Astfel vei înţelege că versetele se referă la oameni care doar pretind că sunt creştini sau oameni care au cunoscut calea mântuirii şi au refuzat-o. Nicăieri în Biblie nu se afirmă că un credincios născut din nou se poate pierde.
	
\end{enumerate}
