\newpage

\section*{Lecția 1}

\subsection*{Dovezi ale nașterii din nou}

Ce se întâmplă cu o persoană care a fost născută din nou? În primul rând, acel om conştientizează că este păcătos. Doar Duhul Sfânt al lui Dumnezeu îl poate face să înţeleagă aceasta. El înţelege că Dumnezeu este sfânt şi trebuie să pedepsească păcatul. Pedeapsa pentru păcat este izgonirea pentru totdeauna din prezenţa lui Dumnezeu într-un loc numit Iad. Această persoană recunoaşte în faţa lui Dumnezeu că este păcătoasă şi că are nevoie de un Mântuitor. Ea înţelege că Dumnezeu L-a dat pe Fiul Său, Domnul Isus Hristos, ca să moară în locul păcătosului. Când această persoană cere ca Domnul Isus Hristos să fie Mântuitorul şi Domnul ei, păcatele îi sunt iertate. În acel moment este născut din nou şi devine un copil în familia lui Dumnezeu.

Păcătosul este acum un credincios în Domnul nostru Isus Hristos şi este numit creştin. Dar acesta este doar începutul noii sale vieţi. Biblia ne spune că au loc mult mai multe lucruri minunate. Iată zece dintre ele:


\begin{enumerate}
	\item Dumnezeu îl acceptă pe creştin pentru că acesta L-a ales pe Domnul Isus Hristos ca să fie Mântuitorul Lui. Priveşte în Efeseni 1:3,6. Dumnezeu, Tatăl, ne-a acceptat în Preaiubitul Lui. Preaiubitul este Domnul Isus Hristos, Fiul lui Dumnezeu, pe care Îl iubeşte. Dumnezeu priveşte mai întâi la Fiul Lui şi apoi la creştin. El îl vede pe creştin ca pe unul care aparţine Fiului Său şi Îl primeşte aşa cum Îl primeşte şi pe Fiul Lui. Dumnezeu se bucură să-L aibă pe creştin în prezenţa Sa. Creştinul este binevenit în prezenţa lui Dumnezeu atâta timp cât este binevenit şi Hristos, adică întotdeauna (Efeseni 2:6,7).

\item Un creştin este un copil al lui Dumnezeu (Ioan 1:12). Cât de onorat te-ai simţi dacă ai fi fiul unui mare conducător, rege sau preşedinte! Cu cât mai mare este onoarea de a fi copilul puternicului Creator al tuturor lucrurilor! Aparţii familiei Lui şi Îl poţi numi pe Dumnezeu Tată.

\item Creştinul este justificat (considerat neprihănit) de Dumnezeu (Romani 5:1 şi 8:30,33). A fi justificat înseamnă mai mult decât a fi iertat. Un om care a comis o crimă poate fi iertat, dar nejustificat. Gândeşte-te la un caz ca acesta: Un om a furat bani de la vecinul lui şi a fost adus în faţa judecătorului. Omul este dovedit vinovat şi merită să fie trimis la închisoare, dar chiar în acel moment în sala de judecată apare un prieten de-al lui. Acest prieten spune: „Voi plăti tot ce datorează acest om şi de asemenea amenda care i se cuvine pentru furt, dacă îl veţi elibera.” Judecătorul este de acord să facă asta iar omului îi este permis să iasă din sala de judecată, eliberat de pedeapsa păcatului său, deşi el este încă vinovat. Este iertat şi nu mai este pedepsit pentru că altcineva a plătit pedeapsa pentru el, însă nu este justificat. Actul lui păcătos rămâne, iar lumea îşi va aminti că a fost un hoţ.

Dumnezeu, care este Judecătorul, afirmă că cel credincios în Hristos este NEVINOVAT. Pedeapsa pentru păcatul lui este moartea (Romani 6:23). Dar Hristos a murit în locul credinciosului. Moartea Lui a plătit pedeapsa pentru păcatul credinciosului. Dumnezeu l-a iertat, iar el nu va fi niciodată judecat pentru greşeala lui (Efeseni 1:7 şi Ioan 5:24). Însă Dumnezeu îl şi justifică pe credincios. Aceasta înseamnă că păcatele lui sunt îndepărtate complet, aşa încât Dumnezeu îl vede ca un om nou, fără o viaţă trăită în păcat. Dumnezeu se uită la el ca şi cum nu ar fi păcătuit niciodată.

\item Trupul credinciosului este locuinţa Duhului Sfânt (1 Corinteni 6:19). Biblia ne învaţă că Dumnezeu Duhul Sfânt trăieşte în fiecare persoană născută din nou (1 Ioan 4:13). Trupul credinciosului este templul sau casa în care Duhul Sfânt locuieşte. Credinciosul ar trebui să fie atent la ceea ce face, spune şi la locurile în care merge, pentru că Duhul Sfânt locuieşte în el. 

\item Creştinul este membru al adevăratei Biserici. Biserica este alcătuită din toţi credincioşii în Domnul Isus Hristos (Fapte 2:47). Nu este onoare mai mare pe pământ decât să aparţii adevăratei Biserici. Această Biserică nu poate fi văzută pentru că este alcătuită din credincioşi din lumea întreagă. Ea este descrisă ca fiind „Trupul lui Hristos” (Coloseni 1:18). „Trupul” este pe pământ, iar Hristos, „Capul Bisericii”, este în ceruri. 

\item Creştinul este moştenitor al lui Dumnezeu (Romani 8:17). Un moştenitor este o persoană care primeşte proprietatea unei alte persoane care a murit. De exemplu, un om care avea mult pământ şi sume imense de bani în bancă moare. După moartea lui, aceste lucruri sunt împărţite între fiii lui. Ei sunt moştenitorii lui şi primesc toate lucrurile pe care tatăl lor le-a deţinut. Dumnezeu deţine toate lucrurile. El a făcut lumea în care trăim şi corpurile cereşti – soarele, luna şi stelele. Dumnezeu este Creatorul universului. Dumnezeu nu are sfârşit, dar într-o zi El va împărtăşi din toate comorile Sale cu toţi cei care sunt născuţi din nou, pentru că ei aparţin lui Hristos. De aceea creştinul este numit moştenitor al lui Dumnezeu.

\item Creştinul este numit sfânt (Romani 1:7). Biblia îi numeşte pe cei care sunt mântuiţi, sfinţi sau poporul lui Dumnezeu. Un sfânt este un om pe care Duhul Sfânt l-a ales să aparţină lui Dumnezeu. Duhul Sfânt îl pune deoparte pe credincios pentru a fi o persoană specială pentru Dumnezeu. Dumnezeu priveşte la creştin prin Fiul Său. El Îl vede mai întâi pe Hristos, iar Hristos este sfânt. Hristos nu a păcătuit niciodată, El a fost sfânt mereu. Astfel Dumnezeu îl vede pe creştin îmbrăcat în sfinţenia lui Hristos (1 Corinteni 1:2). În felul acesta creştinul este o persoană foarte specială, pusă deoparte pentru Dumnezeu. El este sfânt. 

\item Creştinul este desăvârşit în Hristos, el are totul deplin în El (Coloseni 2:10). Aceasta înseamnă că nu mai poate fi adăugat nimic pentru ca creştinul să fie acceptat înaintea lui Dumnezeu. Oamenii pot vedea în creştin multe lucruri imperfecte şi care nu sunt la standardul lui Dumnezeu. Dar Hristos a atins toate standardele lui Dumnezeu, El este perfect. Dumnezeu îl vede pe credincios în Hristos, ceea ce înseamnă că El priveşte mai întâi la Hristos şi este satisfăcut. Apoi Dumnezeu priveşte la credincios prin Hristos şi este de asemenea satisfăcut, iar asta pentru că cel credincios este în Hristos, care este absolut sfânt. În felul acesta credincioşii sunt desăvârşiţi în Hristos.

\item Creştinul a primit o natură dumnezeiască (2 Petru 1:4). Natura divină este natura lui Dumnezeu. Fiecare persoană îşi primeşte natura omenească de la părinţii lui. Aceasta este firea sau natura păcătoasă a lui Adam, care este transmisă fiecărui membru al rasei umane. (Vezi ,,Ce ne învaţă Biblia'' – Lecţia 3). Creştinul are două naturi – natura lui umană primită prin naştere şi natura divină primită atunci când este născut din nou. Natura cea divină, nouă, îl determină să urască păcatul şi să îşi dorească să cunoască mai mult din Dumnezeu. El înţelege astfel că are o nouă dragoste pentru Dumnezeu şi pentru ceilalţi creştini. Natura divină din credincios îl face tot mai asemănător Domnului Isus Hristos (Coloseni 3:10 şi 2 Corinteni 3:18). Aceasta este dorința lui Dumnezeu cu privire la fiecare din copiii Lui.

\item Creştinul se bucură ştiind că Dumnezeu se îngrijeşte de el (Romani 8:28). Dumnezeu are cunoştinţă de tot ce se întâmplă cu copiii Lui. El permite ca celor credincioşi să li se întâmple doar acele lucruri care sunt spre binele lor. Unele lucruri aduc bucurie, dar în viaţa creştinului vin şi probleme şi încercări care au menirea de a-l învăţa mai mult din dragostea lui Dumnezeu. Aceste lucruri îl aduc pe credincios mai aproape de El. În felul acesta Dumnezeu Îşi învaţă copiii să fie răbdători şi să se încreadă în El chiar şi atunci când nu Îi înţeleg căile (Romani 5:3-5).

\end{enumerate}

Acestea sunt doar câteva din lucrurile minunate pe care le-a făcut Dumnezeu. Ele ar trebui să-l determine pe creştin să-L iubească pe Dumnezeu. Dragostea pentru Dumnezeu poate fi arătată prin:

\begin{enumerate}
	\item Închinare şi laudă la adresa lui Dumnezeu pentru mântuirea prin Domnul Isus Hristos. Închinarea înseamnă a-I mulţumi lui Dumnezeu şi a-L iubi.
	
	\item Slujirea lui Dumnezeu cu o inimă binevoitoare. Subiectul slujirii lui Dumnezeu va fi tratat în Lecţia 12.
	
\end{enumerate}

\vspace{2cm}

\nota{După ce ai studiat bine această lecţie, fă testul de auto-evaluare. Apoi răspunde la prima parte din Testul de verificare 1. Răspunde la întrebările 1-10 cu privire la Lecţia 1.}