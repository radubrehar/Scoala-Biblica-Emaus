\newpage

\section*{Lecția 6}

\subsection*{Îngropat în botez}

Ce este botezul şi cine ar trebui să fie botezat? Înainte ca Domnul Isus să meargă în ceruri, El le-a spus ucenicilor Săi: \textit{Duceţi-vă şi faceţi ucenici din toate neamurile, botezându-i în Numele Tatălui şi al Fiului şi al Sfântului Duh. Şi învăţaţi-i să păzească tot ce v-am poruncit. Şi iată că Eu sunt cu voi în toate zilele, până la sfârşitul veacului} (Matei 28:19,20).

Domnul a dorit ca ucenicii Lui să meargă la oameni din orice naţiune a lumii şi să le spună despre moartea şi învierea Lui. El a vrut ca întrega lume să ştie că orice om din orice colţ al lumii poate fi mântuit. Aceasta este Evanghelia sau Vestea Bună. De asemenea, Domnul a vrut ca cei care cred Evanghelia să fie botezaţi. Aceasta ridică două întrebări. În primul rând, cum trebuie efectuat botezul? În al doilea rând, care este semnificaţia botezului?

Pentru a găsi răspunsul la prima întrebare vom deschide la Faptele Apostolilor 8:26-39. Acolo citim despre un om care a fost botezat chiar după ce a fost mântuit. Acest om era un important trimis oficial, un famen responsabil cu finanţele Împărătesei Etiopiei. El fusese la Ierusalim şi se întorcea acasă. În timp ce călătorea în carul lui pe drumul pustiu şi nu prea populat, citea din Vechiul Testament. Dumnezeu ştia cât de mult îşi dorea acest om să cunoască semnificaţia cuvintelor pe care le citea. 

În acelaşi timp, Duhul Sfânt i-a spus unui predicator pe nume Filip să meargă în locul în care îl va întâlni pe acel bărbat. Filip a ascultat şi pe măsură ce se apropia de car, l-a auzit pe acel om citind cu voce tare din cartea Isaia, capitolul 53. Filip i-a explicat că versetele din care citea se refereau la Domnul Isus. I-a spus că Domnul Isus a murit pe cruce pentru ca cei păcătoşi să poată fi mântuiţi. Omul a crezut în Domnul Isus, după care l-a întrebat pe Filip dacă poate fi botezat. Filip a fost de acord să-l boteze, deoarece a crezut cu adevărat în Hristos. Carul a fost oprit lângă o apă. În Fapte 8:38,39 citim cum a fost botezat – „\textit{...s-au coborât amândoi în apă, şi Filip a botezat pe famen.}” Apoi citim că au ieşit afară din apă. Din aceste versete putem observa că persoana botezată a fost îngropată în apă pentru un moment. Acesta este un simbol sau o imagine a ceea ce vom discuta pe măsură ce vom răspunde la a doua întrebare.

Care a fost semnificaţia acestui botez realizat atât de simplu pe marginea unui drum, cu mulţi ani în urmă? Ce semnificaţie are astăzi botezul?

\begin{enumerate}

	\item Botezul este un act de ascultare a poruncii Domnului Isus (Matei 28:19). Botezul nu îndepărtează păcatul. Păcatul nu poate fi spălat cu apă. \textit{Sângele lui Isus Hristos, Fiul lui Dumnezeu este singurul care ne curăţeşte de orice păcat} (1 Ioan 1:9). Trebuie botezaţi doar cei care au crezut în Domnul Isus (Fapte 8:18). Atunci când creştinul se supune voii Domnului şi este botezat, are un cuget bun, curat şi plin de pace înaintea lui Dumnezeu (1 Petru 3:21).
	
	\item Botezul este şi un simbol sau o imagine a unui adevăr spiritual (Romani 6:3-5).
	
	\begin{enumerate}
		\item Uneori Biblia foloseşte apa ca o imagine a judecăţii şi a morţii. 
		\item Când Hristos a murit pe cruce, toate valurile judecăţii şi morţii au venit peste El, păcatele noastre fiind astfel îndepărtate (Psalmul 42:7).
		
		\item Credinciosul a murit împreună cu Domnul Isus pentru că Hristos a murit în locul credinciosului. Când Hristos a murit, credinciosul a murit şi el, când Hristos a fost îngropat, şi credinciosul a fost îngropat, când Hristos a înviat din morţi, credinciosul a înviat şi el din morţi la o viaţă nouă. Botezul este o imagine a morţii, îngropării şi învierii credinciosului. El este scufundat sub apă (îngropare) şi iese din nou afară pentru a trăi o viaţă nouă pentru Dumnezeu.
		
		\item Dumnezeu îl vede pe credincios ca unul care a fost înviat din morţi cu Hristos. Viaţa pe care Dumnezeu o vede în credincios este viaţa lui Hristos (Galateni 2:20). Creştinul a murit faţă de păcat şi faţă de tot ce a fost înainte de a fi mântuit. Dumnezeu nu mai vede omul vechi, ci omul nou în Hristos Isus.
		
		\item Atunci când se botează, un creştin le permite celorlalţi să vadă că aparţine lui Hristos. El s-a unit cu Hristos în moarte, îngropare şi înviere (Coloseni 2:12 şi 3:1,2). Din acel moment, creştinul ar trebui să trăiască o viaţă nouă. Atunci când a fost botezat, el a mărturisit înaintea prietenilor lui că firea lui veche şi iubitoare de păcat a fost îngropată cu Hristos. Măreaţa putere pe care Dumnezeu Tatăl a folosit-o ca să-L învieze pe Hristos a folosit-o şi ca să-i dea o nouă viaţă creştinului. Acum Dumnezeu vrea ca el să arate această nouă viaţă (Romani 6:4).
	\end{enumerate}
	
	\item O persoană care s-a botezat arată prin viaţa ei că vechea ei natură, cea pământească a fost omorâtă. Botezul este un ritual exterior care este de folos doar dacă natura veche este considerată moartă, iar cel botezat trăieşte noua viaţă în Hristos. 
	
\end{enumerate}

Cu mulţi ani în urmă, un credincios care fusese botezat a fost deseori tulburat şi apoi condamnat la moarte. Dar când alţi oameni au fost mântuiţi, şi ei au vrut să fie botezaţi. Aceşti oameni care au fost botezaţi au luat locul celor care au fost ucişi (1 Corinteni 15:29). În unele ţări botezul generează multă mânie din partea păcătoşilor faţă de credincioşi. Uneori, când aceştia Îl mărturisesc pe Hristos doar prin cuvinte, sunt lăsaţi în pace, dar când Îl mărturisesc pe Hristos şi prin botez, duşmanii lui Hristos încep o luptă crâncenă împotriva lor. Credinciosul care este botezat poate avea probleme, dar de asemenea el are bucuria pe care a avut-o famenul etiopian. Citim despre el că îşi vedea de drum plin de bucurie (Faptele Apostolilor 8:39).
\newline
\newline
\textit{Dacă ştiţi aceste lucruri, ferice de voi dacă le faceţi!} (Ioan 13:17).
