\newpage

\lectia{8}{Planul lui Dumnezeu pentru viaţa noastră}

Cum poate cunoaşte un creştin voia lui Dumnezeu în viaţa lui? Are Dumnezeu un plan pentru viaţa noastră? Dacă da, cum putem afla care este acest plan? Multe versete din Biblie ne arată că Dumnezeu are un plan cu viaţa noastră – vom studia câteva din aceste versete şi de asemenea modul în care putem cunoaşte planul Lui special pentru noi.

Fiecare creştin ar trebui să fie interesat să cunoască voia lui Dumnezeu cu privire la el. Vieţile noastre sunt risipite dacă nu cunoaştem şi nu ne supunem planului Domnului. Biblia ne învaţă că cei care Îl recunosc pe Dumnezeu în toate căile lor punându-L pe primul loc în toate vor fi călăuziţi de El şi li se va arăta calea cea bună (Proverbe 3:5,6).

Planul lui Dumnezeu pentru noi este deseori numit „voia lui Dumnezeu”, referindu-se la modul în care El vrea ca noi să trăim. Fiecare credincios poate cunoaşte voia lui Dumnezeu (Romani 12:2). Sunt cinci paşi pe care îi putem urma în vederea cunoaşterii voii lui Dumnezeu. Aceştia sunt: Predare, Mărturisire, Rugăciune, Studiu, Aşteptare.

\begin{enumerate}

	\item \textsl{Predare}. \\ 
	
	Trebuie să fim gata să dăm la o parte propriile noastre speranţe şi lucrurile pe care vrem să le facem. Trebuie să ne dorim calea lui Dumnezeu mai mult decât orice altceva şi să fim dispuşi să renunţăm la calea noastră. Pavel s-a predat Domnului atunci când a întrebat: \textit{Doamne, ce vrei să fac?} (Fapte 9:6). Isaia şi Amasia au trăit în timpul Vechiului Testament. Isaia s-a predat atunci când a spus: \textit{Iată-mă, trimite-mă} (Isaia 6:8) şi mai citim că Amasia s-a predat de bună voie Domnului (2 Cronici 17:16). Astăzi, creştinii trebuie să facă aşa cum a făcut Amasia – Dumnezeu ne cere să ne predăm Lui. În Romani 12:1,2 suntem îndemnaţi să ne aducem pe noi înşine ca o jertfă vie lui Dumnezeu. Trebuie să fim gata să facem asta pentru marea îndurare pe care Dumnezeu a arătat-o faţă de noi. De asemenea ni se spune \textit{să nu ne potrivim chipului veacului acestuia} sau să copiem comportamentul oamenilor nemântuiţi. Dacă ne predăm lui Dumnezeu, El va transforma mintea noastră pentru a fi ca a lui Hristos. În felul acesta vom putea cunoaşte voia lui Dumnezeu – vom cunoaşte ce este bine şi plăcut înaintea Lui.
	
	\item \textit{Mărturisire} \\
	
	Pentru a cunoaşte voia lui Dumnezeu pentru noi, trebuie să ne mărturisim păcatele la care poate nu vrem să renunţăm şi să ne lăsăm de ele. Scriitorul Psalmului 66:18 spunea: \textit{Dacă aş fi cugetat lucruri nelegiuite în inima mea, nu m-ar fi ascultat Domnul}. De asemenea, ar trebui să ne mărturisim slăbiciunile şi să ne încredem în puterea lui Dumnezeu de a ne ajuta (Psalmul 139:23,24 şi Ieremia 10:23).  
	
	\item \textit{Rugăciune} \\
	
	Trebuie să venim înaintea Domnului în mod regulat şi să Îi cerem să ne călăuzească. Trebuie să ne bazăm pe promisiunea Lui că ne va învăţa şi ne va îndruma şi astfel să Îi cerem să facă aşa cum a promis. Pe măsură ce ne rugăm, să ne dorim slava Lui mai presus de orice altceva (Coloseni 1:9 şi 4:12).
	
	\item \textit{Studiu} \\
	
	Petrece mult timp citind Cuvântul lui Dumnezeu. Citeşte-l rar şi meditează la el. Memorează versete din Scriptură. Citeşte Cuvântul în timp ce eşti pe genunchi şi te rogi lui Dumnezeu să-ţi vorbească prin el (Psalmul 143:8,10). Studiază-l comparând un verset cu altele.
	
	\item \textit{Așteptare} \\
	
	Dacă Dumnezeu nu răspunde imediat, aşteaptă (Psalmul 62:5). Dacă nu primeşti niciun răspuns cu privire la un anumit lucru, acceptă această tăcere ca fiind călăuzirea lui Dumnezeu şi pentru moment nu fă acel lucru. Dacă într-adevăr te încrezi în Domnul, nu te vei grăbi. Cel ce Îl are ca sprijin pe Domnul, nu se va grăbi (Isaia 28:16). 
	
\end{enumerate}

Domnul ne descoperă voia Lui în diferite moduri. El poate folosi una sau mai multe din următoarele metode:

\begin{enumerate}

	\item Călăuzire prin Biblie. Scriptura ne călăuzeşte în două moduri:
	
	În primul rând, ea ne vorbeşte în mod clar despre anumite lucruri pe care nu trebuie să le facem. De exemplu, dacă un creştin se roagă pentru călăuzire cu privire la căsătoria cu o persoană nemântuită, el poate primi răspunsul lui Dumnezeu în 2 Corinteni 6:14, unde citim: \textit{Nu vă înjugaţi la un jug nepotrivit cu cei necredincioşi. Căci ce legătură este între neprihănire şi fărădelege? Sau cum poate sta împreună lumina cu întunericul?} 

	În al doilea rând, Dumnezeu foloseşte deseori un verset sau mai multe versete din Scriptură pentru a ne călăuzi să facem anumite lucruri. Un verset pe care înainte nu-l observasem poate dobândi o nouă semnificaţie deoarece ne spune ce să facem chiar în momentul în care ne rugăm pentru călăuzire (Psalmul 119:105).

	\item Călăuzire prin alţi credincioşi. Uneori ne este de ajutor să cerem sfatul unor creştini experimentaţi care cunosc Cuvântul lui Dumnezeu. Experienţa şi sfatul lor îl pot scuti pe credincios de multe probleme. El ar trebui să asculte şi să urmeze sfatul lor.
	
	\item Călăuzire prin împrejurări. Dumnezeu controlează toate lucrurile şi uneori ne arată voia Lui prin lucrurile pe care le îngăduie în viaţa noastră. De exemplu, o mamă se roagă pentru un fiu care este bolnav într-un alt oraş. Ea se întreabă dacă ar trebui să meargă la el, în condiţiile în care nu are bani pentru a plăti biletul până acolo. Dacă între timp primeşte o scrisoare cu banii de care are nevoie, ea este sigură că Dumnezeu se foloseşte de aceasta ca să-i arate că ar trebui să meargă la fiul ei.
	
	\item Călăuzire prin Duhul Sfânt. Duhul Sfânt al lui Dumnezeu poate controla gândirea şi dorinţele noastre aşa încât voia lui Dumnezeu să fie clară pentru noi. Când luăm o decizie care este în acord cu voia lui Dumnezeu avem pace lăuntrică. Aceasta ne asigură că Îi suntem plăcuţi lui Dumnezeu (Coloseni 3:15). Această călăuzire este atât de clară încât a o refuza ar însemna să fim neascultători. 
	
\end{enumerate}

Încă un cuvânt – când Dumnezeu ne luminează, trebuie să acţionăm ca atare (Faptele Apostolilor 26:19). Călăuzirea lui Dumnezeu trebuie urmată dacă vrem ca Domnul să continue să ne-o ofere. O viaţă de bucurie adevărată este bazată pe ascultarea de Dumnezeu. 
