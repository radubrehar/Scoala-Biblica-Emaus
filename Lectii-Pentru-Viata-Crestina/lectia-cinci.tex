\newpage

\section*{Lecția 5}

\subsection*{Comportamentul creştinului}

Cum poate şti un creştin ce îi este sau ce nu îi este îngăduit să facă? Este în regulă dacă face toate lucrurile pe care cei nemântuiţi le fac pentru a se distra? Ar fi greşit ca un creştin să facă anumite lucruri? Cum rămâne cu petrecerile în cadrul cărora se consumă băuturi alcoolice? Dar cu dansul sau filmele? Biblia vorbeşte despre anumite lucruri pe care nu trebuie să le facem, dar nu ne spune despre toate lucrurile. Cum putem şti ce este corect?

Această lecţie îţi va oferi doisprezece întrebări pe care să le foloseşti atunci când trebuie să decizi dacă  o activitate este bună sau greşită.

\begin{enumerate}

	\item Afirmă Biblia cu claritate că Domnul interzice lucrul de care nu eşti sigur dacă trebuie să-l faci? Nu face lucrul respectiv până ce nu ai ocazia să afli ce a spus Dumnezeu despre asta. Dacă găseşti că El spune să nu-l faci, fereşte-te de acest lucru ca de o boală mortală (1 Tesaloniceni 5:22).
	
	\item Îi aduce acest lucru vreo slavă lui Dumnezeu? Va fi Dumnezeu onorat sau lăudat de ceea ce faci tu? În 1 Corinteni 10:31 citim „...fie că mâncaţi, fie că beţi, fie că faceţi altceva: să faceţi totul pentru slava lui Dumnezeu.” Poţi cu adevărat să ceri binecuvântarea lui Dumnezeu peste lucrul respectiv înainte de a lua parte la el? Va fi Dumnezeu onorat?
	
	\item Aparţine lumii? Când Domnul Isus S-a rugat pentru ai Lui, El a spus că aceştia nu erau din lume tot aşa cum nici El nu era din lume (Ioan 17:14-16). Deşi a trăit în lume, El nu a făcut parte din lumea care L-a respins pe Dumnezeu. Ceea ce a spus El despre Dumnezeu şi despre om a fost foarte diferit de gândurile altor oameni. El nu a aparţinut lumii şi nici creştinii nu aparţin lumii. Dacă lucrul despre care te întrebi este ceea ce face lumea şi este diferit de adevărurile pe care le-ai învăţat din Biblie, atunci aparţine lumii şi nu lui Hristos (1 Ioan 2:15-17).
	
	\item Ar face Domnul Isus lucrul de care tu nu eşti sigur? El ne-a lăsat un exemplu, iar noi trebuie să călcăm pe urmele paşilor Lui (1 Petru 2:21).
	
	\item Ţi-ar plăcea ca Domnul să te găsească făcând acel lucru când Se va întoarce? Nu fă niciun lucru, nu te duce în niciun loc şi nu spune nimic care ţi-ar aduce ruşine în cazul în care Domnul ar apărea dintr-o dată (1 Ioan 2:28).
	
	\item Continui să-ţi doreşti să faci acel lucru atunci când îţi aminteşti că Dumnezeu Duhul Sfânt locuieşte în tine? Trupul tău este templul Duhului Sfânt care locuieşte în tine, iar tu L-ai primit ca dar de la Dumnezeu. Tu nu mai eşti al tău – trupul tău nu-ţi mai aparţine (1 Corinteni 6:19). Citeşte de asemenea Efeseni 4:30. Domnul a plătit pentru tine cu propriul Lui sânge preţios (1 Petru 1:18,19).
	
	\item Ar fi normal ca tu, în calitate de copil al lui Dumnezeu să acţionezi în felul acesta? Fiul unui rege sau al unui conducător aduce ruşine numelui tatălui său atunci când face ceva greşit. Un creştin aduce ruşine, dezonoare numelui lui Dumnezeu atunci când face lucruri păcătoase (Romani 2:24 şi Coloseni 1:10).
	
	\item Vor putea vedea ceilalţi oameni o diferenţă între un creştin şi un om nemântuit când te vor vedea făcând acel lucru? (2 Corinteni 5:17). L-ar determina pe un alt creştin să urmeze exemplul tău – mergând poate chiar puţin mai departe decât tine, căzând astfel în păcat? Apostolul Pavel a îndemnat ca nimeni să nu pună în calea unui alt creştin un lucru care l-ar putea face să cadă în păcat (Romani 14:13).
	
	\item Există vreo îndoială oricât de mică în mintea ta cu privire la lucrul pe care vrei să-l faci? Dacă da, atunci nu îl fă.
	
	\item Ai putea folosi acel timp într-un mod mai bun? Noi trebuie să răscumpărăm vremea (Efeseni 5:16). Asta înseamnă că ar trebui să folosim timpul în cel mai bun mod cu putinţă. Unele lucruri bune ne pot împiedica să le facem pe \textit{cele mai bune}.
	
	\item Dacă este vorba despre nişte bani pe care urmează să-i cheltuieşti, ar putea fi ei folosiţi într-un mod mai bun? Ar trebui să ne folosim banii în cel mai înţelept mod posibil, pentru slava lui Dumnezeu şi binecuvântarea altora (Luca 16:11).
	
	\item Iar apoi, este acest lucru o piedică? (Evrei 12:1) În acest verset viaţa creştinului este comparată cu o cursă – din momentul în care a fost mântuit până ce ajunge în ceruri. Un atlet din cadrul unei curse care îşi doreşte cu adevărat să câştige, nu poartă la el o mulţime de lucruri de care nu are nevoie. Aşadar un creştin ar trebui să dea drumul acelor lucruri care îl împiedică să fie tot mai asemănător cu Hristos.
	
\end{enumerate}

Gândindu-ne la aceste lucruri, este bine să amintim că noi nu suntem sub lege, ci sub har (Romani 6:14,15). Noi nu stăm deoparte de lucrurile pe care ceilalţi oameni le fac pentru că Dumnezeu ne obligă să facem asta, ci pentru că Îl iubim pe Domnul. Vrem să facem ce Îi face plăcere lui Dumnezeu pentru că El a făcut atât de mult pentru noi. Hristos a murit pentru noi, iar noi vrem să trăim o viaţă plăcută lui Dumnezeu (2 Corinteni 5:14,15). Dumnezeu nu spune „Dacă stai departe de plăcerile păcătoase vei deveni creştin”, ci „Tu eşti creştin! Trăieşte aşa cum ar trebui să trăiască un creştin” (Efeseni 4:1).