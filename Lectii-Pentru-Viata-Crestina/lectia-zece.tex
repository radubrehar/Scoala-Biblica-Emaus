\newpage

\lectia{10}{Conducându-i pe alţii la Hristos}

Cum îi poate conduce un creştin pe alţi oameni la Hristos? Aducerea altora la Isus Hristos este una din cele mai măreţe lucrări care sunt astăzi în lume (Proverbe 11:30). Nu există reguli pentru a avea succes în această lucrare, dar există câteva sfaturi care pot fi de ajutor:

\begin{enumerate}

	\item Creştinul trebuie să citească Cuvântul lui Dumnezeu şi să petreacă mult timp în rugăciune dacă vrea să-i conducă pe alţii la Domnul. Mărturiseşte toate păcatele din viaţa ta. Fii gata să-I predai toate dorinţele tale lui Dumnezeu. În felul acesta, Duhul Sfânt va conduce viaţa ta, iar Domnul îţi va da şansa de a le spune altora cum pot fi mântuiţi. În Matei 4:19 ni se spune să-L urmăm pe Domnul dacă ne dorim să fim câştigători de suflete.
	
	\item Începe fiecare zi cerându-I lui Dumnezeu să te conducă la cei cărora trebuie să le vorbeşti despre Evanghelie. Nu poţi vorbi oricărui om pe care îl vezi. De asemenea, nu ai cum să ştii cine este gata să asculte şi să primească Evanghelia. Va fi o lucrare mult mai uşoară dacă Îl laşi pe Domnul să te călăuzească, astfel mai mulţi oameni vor fi mântuiţi.
	
	\item Ori de câte ori ai ocazia să-i spui unei persoane despre Domnul Isus, ar trebui să faci asta. Când oamenii te întreabă dacă eşti creştin, spune-le asta. Începe să vorbeşti despre Evanghelie. Oamenii sunt dispuşi să vorbească despre jocuri, vreme, politică şi evenimente cotidiene. De ce ţi-ar fi teamă să vorbeşti cu ei despre Domnul Isus Hristos?
	
	\item Citează din Cuvântul lui Dumnezeu cât de mult posibil. Este un Cuvânt viu (Evrei 4:12) care are putere de a le vorbi oamenilor cu mult mai presus decât o pot face cuvintele tale. Este „sabia Duhului” şi fiecare soldat bun al lui Isus Hristos ar trebui să folosească această armă care este cea mai măreaţă dintre toate. Oamenii nemântuiţi vor face tot ce le va sta în putinţă să te oprească din a le citi versete biblice. Nu te opri. Când cei nemântuiţi îţi vor spune că ei nu cred Biblia, citează mai mult din ea.
	
	\item Încearcă să vorbeşti a doua oară cu persoana căreia i-ai spus Evanghelia. Nu mulţi sunt cei care Îl primesc pe Hristos prima oară când aud despre El. De obicei, oamenii au nevoie să li se vorbească din nou şi din nou. Le poţi oferi broşuri care prezintă Evanghelia şi îi poţi invita să vină cu tine într-un loc unde se predică Evanghelia. Uneori celor care te contrazic sau care refuză să te asculte, Duhul Sfânt le va vorbi. Nu fii descurajat, ci fă lucruri mici pentru a-i ajuta pe prietenii tăi ori de câte ori poţi şi continuă să te rogi pentru ei.
	
	\item Nu grăbi pe nimeni să afirme că este mântuit. Dacă spune că este mântuit nu înseamnă că inima lui este schimbată. Adevărata credinţă în Hristos trebuie să vină din inimă. Doar astfel viaţa va fi schimbată. Tu doar spune-le altora Evanghelia cu credincioşie, iar Dumnezeu va folosi cuvintele tale. Lasă rezultatele în mâna Lui.
	
	\item Cere-I lui Dumnezeu ajutor atunci când îţi este greu să le vorbeşti celor nemântuiţi despre Evanghelie. Dumnezeu îţi va da cuvintele potrivite şi curajul de care ai nevoie (Evrei 4:16).
	
	\item Asigură-te că ai mereu la tine un număr mare de broşuri care prezintă Evanghelia. Poţi oferi tractate celor pe care îi întâlneşti sau le poţi lăsa în autobuze, restaurante şi în multe alte locuri. 
	
	\end{enumerate}

	
Răsplăţile pe care le vei primi în urma conducerii altor persoane la Hristos sunt mari.

	\begin{enumerate}
	
		\item Bucuria pe care o ai atunci când aduci o persoană la Hristos este mare (Luca 15:10).
		\item Gândeşte-te câtă bucurie vei avea în ceruri când vei fi salutat de cineva care îţi va spune: „Tu m-ai invitat aici!”   
		\item Domnul Isus Hristos a spus că El va vorbi înaintea tuturor oştirilor cereşti despre cei care au vorbit despre El şi L-au mărturisit pe pământ (Matei 10:32). Ce moment plin de emoţie va fi acela! Gândindu-ne la aceste lucruri, fie ca rugăciunea noastră să fie:\\
		
	Vreau să privesc înspre mulţime aşa cum a privit al meu Mântuitor\\
	Până ce ochii mi se vor îneca în lacrimi\\	
	Vreau să privesc cu milă spre turma de oi rătăcite\\
	Şi să le iubesc de dragul Lui. (Matei 9:36)\\


\end{enumerate}