\newpage

\lectia{11}{Studierea Bibliei}

Ce trebuie să ştie un creştin despre Biblie? Studierea Bibliei trebuie făcută în dependenţă deplină de Duhul Sfânt. El este Învăţătorul nostru, iar noi trebuie să-I cerem mereu să ne călăuzească (Ioan 14:26 şi 16:13). Nu există o modalitate rapidă şi uşoară de a studia Biblia. Este o muncă dificilă pentru oricine. Cu toate acestea, este adevărat că pe măsură ce o studiem mai mult, cu atât ne este mai uşor să învăţăm mai mult din ea. 

Următoarele cărţi te vor ajuta pe măsură ce studiezi Biblia:

\begin{enumerate}

	\item Biblii de studiu. Versiunea Cornilescu şi o traducere modernă vor fi de mare ajutor.
	\item O concordanţă biblică bună. Aceasta te ajută să găseşti un verset atunci când îţi aminteşti doar o parte din el. Cu ajutorul ei poţi căuta un cuvânt pentru care vei găsi referinţele biblice în care este folosit. 
	\item Un dicţionar biblic bun. Acest dicţionar constă într-o listă de cuvinte folosite în Biblie şi care prezintă semnificaţia fiecărui cuvânt explicată în mod clar. 
	
\end{enumerate}

Studentul nu trebuie să fie descurajat dacă nu îşi poate permite un dicţionar biblic sau o concordanţă biblică. Pe măsură ce citeşte Biblia, Duhul Sfânt îl va învăţa. În continuare sunt câteva lucruri care vor fi de ajutor în studierea Scripturii:

\begin{enumerate}

	\item Pune-ţi deoparte un timp stabilit pentru a citi Biblia în fiecare zi. O modalitate bună este să începi cu Matei şi să citeşti întreg Noul Testament. Apoi începe cu Geneza şi citeşte întreaga Biblie. Citeşte cu atenţie pentru a înţelege clar ce spune Biblia.
	
	\item Când ajungi la un cuvânt pe care nu-l înţelegi, caută-l în dicţionarul biblic sau într-un dicţionar obişnuit. Continuă să citeşti dacă nu deţii aceste cărţi. Restul pasajului poate clarifica semnificaţia lui. Nu te opri din citit din cauza unei părţi pe care nu o înţelegi. Vor fi multe lucruri pe care nu le vei înţelege pe deplin, mai târziu cineva îţi va explica aceste părţi mai dificile. Poţi cere ajutor cu privire la cuvinte sau versete pe care nu le înţelegi atunci când trimiţi testele. 
	
	\item Compară versetul pe care îl citeşti cu alte versete. Deseori un verset te va ajuta să înţelegi alte versete. Încearcă să afli ce are de spus Biblia în legătură cu anumite subiecte. Ai grijă să nu construieşti niciodată o doctrină pornind de la un singur verset din Scriptură. Acest lucru s-a întâmplat deseori şi de obicei rezultatul a fost o învăţătură falsă. Biblia nu se contrazice pe sine. De exemplu, nu se poate să citim ceva într-o epistolă scrisă de Pavel şi să descoperim că Petru scrie ceva diferit despre acelaşi subiect. 
	
	\item În timp ce studiezi fiecare capitol din Biblie, răspunde la următoarele întrebări şi astfel vei putea înţelege mai bine Cuvântul lui Dumnezeu.
	
Întreabă-te:

	\begin{enumerate}
	
		\item Ce am învăţat despre Domnul Isus din acest capitol? (Chiar şi versetele din Vechiul Testament vorbesc despre Mântuitorul).
		
		\item Care este adevăratul mesaj al acestui capitol? Pot fi versete care se pot referi la mai multe idei. Încearcă să descoperi cel mai important gând evidenţiat în capitolul respectiv.
		
		\item Ce promisiune făcută de Dumnezeu în acest pasaj aş putea-o lua în mod personal?  
		
		\item Care verset pare să se evidenţieze ca cel mai important pentru mine?
		
		\item Cu privire la ce păcat mă învaţă să mă feresc sau să nu am de-a face cu el?
		
		\item Ce exemplu îmi este dat ca să urmez?
		
		\item Care sunt cele mai dificile versete?
	\end{enumerate}
	
	\item Vorbeşte în fiecare zi cu cineva despre versetele pe care le-ai citit. Asta te va ajuta să îţi aminteşti Cuvântul lui Dumnezeu şi va fi o binecuvântare pentru persoana căreia îi împărtăşeşti gândurile tale (Maleahi 3:16). 
	
	\item Încearcă să memorezi câteva versete pe săptămână. Începe cu versete din Evanghelii cum ar fi Ioan 1:12, Ioan 3:16, Ioan 3:36, Ioan 5:24, Romani 10:9 şi multe altele. Recapitulează toate versetele memorate de mai multe ori până când le ştii foarte bine. Vei descoperi că te vei bucura tot mai mult şi vei putea folosi aceste versete atunci când le vei vorbi altora despre Domnul.
	
	\item Scopul cel mai înalt în studierea Bibliei este de a pune în practică ce ai învăţat. Cuvântul lui Dumnezeu îl îndreaptă pe credincios şi îl face tot mai asemenea Domnului Isus (Ieremia 15:16).                                       
	
\end{enumerate}

Adu-ţi aminte că atunci când studiezi Biblia, studiezi Cuvântul lui Dumnezeu, o Carte veşnică. Lucrurile pe care le înveţi din Cuvânt vor rămâne în inima şi mintea ta pentru totdeauna. \textit{,,Cuvintele Mele nu vor trece.''} (Matei 24:35, 1 Petru 1:23,25).
