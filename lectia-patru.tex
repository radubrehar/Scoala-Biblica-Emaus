\newpage

\section*{Lecția 4}

\subsection*{Biruinţă asupra ispitei}

Cum poate respinge un creştin lucrurile păcătoase? De îndată ce o persoană este mântuită, în interiorul ei începe o luptă sau o bătălie puternică. Aceasta se întâmplă pentru că creştinul încă mai are firea veche pe care a primit-o la naştere. Natura lui omenească şi păcătoasă încearcă să-l atragă în păcat. Creştinul a primit o nouă natură atunci când a fost născut din nou. Noua natură sau fire este viaţa din Dumnezeu care urăşte păcatul şi vrea să facă lucrurile plăcute lui Dumnezeu. Cele două naturi sunt mereu în luptă una împotriva celeilalte (Galateni 5:16,17 şi Romani 8:5-8).

Natura umană mai este numită şi natura sau firea veche, fiind deznădăjduit de rea. Ea nu poate fi îmbunătăţită şi rămâne în creştin până ce acesta va ajunge în ceruri. Dumnezeu a declarat-o vinovată şi a pedepsit-o atunci când Hristos a murit pe cruce. Acum El vrea ca creştinul să se raporteze la ea ca şi cum ar fi moartă. Nu o încuraja! Nu o hrăni! Nu-i oferi şanse! (Romani 13:14).

Noua natură îl determină pe creştin să-şi dorească să facă lucruri bune şi pe măsură ce ascultă de noua lui natură devine tot mai puternic. El ar trebui să încerce să facă lucrurile care Îi plac lui Dumnezeu, aceasta îi va încuraja noua lui natură.

Creştinul va fi ispitit să păcătuiască. Satan va încerca să-l facă pe creştin să se întoarcă la viaţa păcătoasă pe care a trăit-o înainte de a fi mântuit. De asemenea unii din prietenii lui nemântuiţi pot încerca să îl determine să facă lucruri păcătoase. Cum poate avea biruinţă asupra acestor ispite?

Iată câteva lucruri care îţi vor fi de ajutor:

\begin{enumerate}

	\item Roagă-te în fiecare zi! Cere-I lui Dumnezeu să te ajute să ştii cum să acţionezi şi ce să faci atunci când eşti ispitit (Evrei 4:16). Dumnezeu îţi va da putere şi te va ajuta să te opui răului (1 Corinteni 10:13). Dacă vei încerca să faci asta de unul singur vei eşua. 
	
	\item ???????????????
	
	\item Hrăneşte şi încurajează noua ta natură! Înfometează-ţi vechea natură! Fii atent unde mergi, ce asculţi la radio sau ce vizionezi la TV şi ce citeşti (Coloseni 3:5-9). Creştinul trebuie să se gândească la Hristos. Nu te poţi gândi la păcat atunci când te gândeşti la Hristos (Coloseni 3:10-14). Gândurile îndreptate spre Hristos îţi vor da putere să trăieşti o viaţă plăcută lui Dumnezeu. Creştinul I Se va închina lui Hristos pe măsură ce se va gândi la El. Vei deveni asemenea Celui în faţa căruia te închini. 2 Corinteni 3:18 ne spune că creştinii devin asemenea Domnului Isus pe măsură ce privesc la El în oglinda Cuvântului Său. Asta înseamnă că Duhul Sfânt care locuieşte în tine te va schimba pe măsură ce citeşti şi meditezi adânc la slava şi sfinţenia lui Hristos. Duhul Sfânt îl face pe creştin tot mai asemănător cu Hristos.
	
	\item Mărturiseşte-ţi păcatele înaintea lui Dumnezeu. De îndată ce conştientizezi că ai păcătuit, cere-I lui Dumnezeu să te ierte. Ar putea fi vorba de un gând păcătos, de vorbe păcătoase pe care le-ai rostit sau despre o acţiune păcătoasă. Nu aştepta. Mărturiseşte-ţi păcatul imediat înaintea lui Dumnezeu (Proverbe 28:13).
	
	\item Împrieteneşte-te cu alţi creştini şi nu cu oameni cărora le place să păcătuiască (Proverbe 10:16 şi Evrei 10:24,25). Poţi locui sau munci împreună cu oameni nemântuiţi. Lasă-i să vadă că eşti creştin prin lucrurile pe care le spui şi le faci. Nu te alătura lor pe căile păcătoase (Efeseni 5:11).
	
	\item Ocupă-te cu lucrurile Domnului! Deseori lenevia duce la păcat. Oferă-I Domnului trupul tău, ca El să-l folosească cum ştie mai bine (Romani 6:19). Este foarte mult de lucru şi Îl vei avea pe Cel mai bun Stăpân. 
	
	\item Antrenează-ţi trupul şi păstrează-l sănătos (1 Timotei 4:8). Trupul tău este casa sau templul Duhului Sfânt care locuieşte în tine (1 Corinteni 6:19,20). Sportul este bun, dar nu ar trebui să ia prea mult din timpul tău îngrădind astfel lucrurile spirituale.
\end{enumerate}

Un creştin nu este scutit de ispitiri dacă biruieşte asupra unei ispite. Ai biruinţă doar dacă depinzi de Dumnezeu pentru ea. Trebuie să depinzi de Dumnezeu în fiecare moment al zilei. Dacă uiţi de Domnul şi neglijezi rugăciunea şi studierea Bibliei, poţi cădea în ispită. Aceasta i se poate întâmpla chiar şi unei persoane care cunoaşte bine Biblia şi este credincioasă de multă vreme.

Îndreaptă-ţi privirea înspre Domnul! (Coloseni 3:1-4).
